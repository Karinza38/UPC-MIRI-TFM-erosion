%---------------------------------------------------------------------------
% Abstract

\chapter*{Abstract}
 \addcontentsline{toc}{chapter}{Abstract}

Synthetic terrains play a vital role in various applications, including entertainment, training, and simulation. While significant progress has been made in terrain generation, existing methods often focus on large-scale features, relying on 2D elevation maps to model them. However, rocky terrains like those found in alpine environments have many detail features like sharp ridges, loose blocks or overhangs that are poorly represented in this maps, so it is common to model them using textures. 

\vspace{0.5\baselineskip}
Instead, in this project, we aim to generate plausible rocky geometry on top of existing 3D models. We propose a method based on a simplified simulation of mechanical erosion processes commonly found in high altitude terrains such as percolation and freeze-thaw weathering. The process can be controlled through a series of intuitive parameters and its iterative nature lets an artist apply it multiple times until sufficient erosion is achieved.

\vspace{0.5\baselineskip}
Additionally, we developed an artist-friendly tool integrated as add-on into Blender, which is a widely used 3D modeling software. This rich integration streamlines their workflow, eliminating the need for external applications and facilitating direct interaction with the model geometry before and after the simulated erosion.

\vspace{0.5\baselineskip}
\begin{itemize}
    \item \textbf{Keywords}: Computer Graphics, 3D Modelling, Computer Simulation, Rocky Terrains, Mechanical Weathering.
    % \item \textbf{CATALAN}: Gràfics per ordinador, Modelatge 3D, Simulació per ordinador, Terrenys Pedregosos, Erosió Mecànica.
\end{itemize}

 %\cleardoublepage

%---------------------------------------------------------------------------
% Acknowledgements

\chapter*{Acknowledgements}
 \addcontentsline{toc}{chapter}{Acknowledgements}

\texttt{I would like to thank my supervisors, Oscar and Toni, for their continuous support and valuable feedback.}

\vspace{0.5\baselineskip}
\texttt{I am also thankful to all my family and friends for always being there for me, specially Miriam, your personal support is truly invaluable.}

 %\cleardoublepage

%---------------------------------------------------------------------------
% Table of contents

 \setcounter{tocdepth}{4}
 \tableofcontents

 %\cleardoublepage

%---------------------------------------------------------------------------
% Symbols

\chapter*{Acronyms and Abbreviations}\label{chap:symbole}
 \addcontentsline{toc}{chapter}{Acronyms and Abbreviations}


\begin{tabbing}
 \hspace*{3.2cm}  \= \kill

%write abbreviations like this\\
2D\> Two-dimensional\\
3D\> Three-dimensional\\
AABB\> Axis-Aligned Bounding Box\\
API\> Application Programming Interface\\
BFS\> Breadth-first search\\
CGI\> Computer-Generated Imagery\\
CPU\> Central Processing Unit\\
CSG\> Constructive solid geometry\\
DEM\> Digital Elevation Model\\
GPU\> Graphics Processing Unit\\
UI\> User Interface\\
VR\> Virtual Environment\\

\end{tabbing}

%\cleardoublepage

%---------------------------------------------------------------------------
